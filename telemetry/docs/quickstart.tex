\documentclass[11pt]{article}
\usepackage{graphicx}
\usepackage{hyperref}
\usepackage{xcolor}
\usepackage{listings}

\newcommand{\todo}[1]{\textcolor{red}{\textbf{TODO:} #1}}

% lifted from http://timmurphy.org/2014/01/27/displaying-code-in-latex-documents/
\lstset{
    basicstyle=\ttfamily\footnotesize,
    breaklines=true,
    frame=tb, % draw a frame at the top and bottom of the code block
    tabsize=2, % tab space width
    showstringspaces=false, % don't mark spaces in strings
    numbers=left, % display line numbers on the left
    commentstyle=\color{green}, % comment color
    keywordstyle=\color{blue}, % keyword color
    stringstyle=\color{red} % string color
}

\title{\textbf{Telemetry Quick Start \& Reference}}
\author{Ducky}
\date{\today}
\begin{document}

\maketitle

\section{Introduction}
The telemetry library is designed to provide a simple and efficient way to get data off an embedded platform and both visualized and logged on a PC. The intended use case is streaming data from an embedded system to a PC through an UART interface (possibly on top of either a USB or Bluetooth transport layer). Since data definitions are transmitted, user-side code changes can be limited to the data source side, eliminating the need to keep both the source-side and sink-side code in sync.

The server-side (data source) code is written in C++ and designed with embedded constraints in mind (mainly, no dynamic memory allocation is done). Templated classes are used to support most native data types and a hardware abstraction layer allows multiple platforms (including Arduino and mBed) to be supported.

The client-side (data sink) code is written in Python. A basic telemetry protocol parser is provided that separates and interprets telemetry packets and other data from the received stream. A matplotlib-based GUI is also provided on top of the parser that visualizes numeric data as a line plot and array-numeric data as a waterfall / spectrogram style plot.

A protocol spec is also given allowing other implementations of either the server and client. It short, it defines a binary wire format along with packet structures for data and headers.

\section{Prerequisites}
The following software is necessary for a basic telemetry install:
\begin{itemize}
  \item \href{https://www.python.org/downloads/}{Python} (this document was written for 3.4, but may work on 2.7)
\end{itemize}

The following software is necessary for the telemetry plotter GUI:
\begin{itemize}
  \item \href{http://sourceforge.net/projects/numpy/files/NumPy/}{NumPy (1.9.2 or later)}
  \begin{itemize}
    \item \href{http://sourceforge.net/projects/numpy/files/NumPy/1.9.2/numpy-1.9.2-win32-superpack-python3.4.exe/download}{executable for Windows, Python 3.4 32-bit}
  \end{itemize}
  \item \href{http://matplotlib.org/downloads.html}{matplotlib (1.4.3 or later)}
  \begin{itemize}
    \item \href{http://sourceforge.net/projects/matplotlib/files/matplotlib/matplotlib-1.4.3/windows/matplotlib-1.4.3.win32-py3.4.exe/download}{executable for Windows, Python 3.4 32-bit}
  \end{itemize}
\end{itemize}

\subsection{Windows Install}
\begin{enumerate}
  \item Install the prerequisite software above.
  \item Install the required Python packages via \texttt{pip}:
  \begin{verbatim}
    cd C:\Python34\Scripts
    pip install pyserial 
  \end{verbatim}
  \item If using the plotter GUI, also install these Python dependencies for matplotlib via \texttt{pip}:
  \begin{verbatim}
    cd C:\Python34\Scripts
    pip install six python-dateutil pyparsing
  \end{verbatim}
\end{enumerate}

\subsection{Linux Install}
\todo{to be written}

\section{Known Issues}
\subsection{Telemetry framework}
\begin{itemize}
  \item No CRC support yet.
  \item Packets do not time out on the receiver side yet.
  \item Receivers can't request that the header packet be re-sent.
\end{itemize}

\subsection{Plotter GUI}
\begin{itemize}
  \item Waterfall / spectrogram-style plotting is highly inefficient (but serviceable).
\end{itemize}

\subsection{mBed HAL}
\begin{itemize}
  \item MODSERIAL and Serial putc is highly inefficient for multi-byte data, having high overhead (at 1 Mbaud, the overhead is about 3x the time to transmit the byte). Using MODDMA may increase efficiency.
\end{itemize}

\subsection{Arduino HAL}
\begin{itemize}
  \item None yet
\end{itemize}

\section{Transmit-side Quick Start Guide}
All the transmit-side code is contained in (repository base)\texttt{/server-cpp}.

\subsection{Build system setup}
\begin{itemize}
  \item Add all the \texttt{.h} header files to your compiler's include path.
  \item Add all the \texttt{.cpp} source files to your build path - these should end up compiled into your code.
\end{itemize}

\subsection{Code setup}
\begin{itemize}
  \item Include the telemetry header in your code:
  \begin{lstlisting}[language=C++]
#include "telemetry.h"
  \end{lstlisting}
  \item Include the HAL for your platform: either \texttt{telemetry-arduino.h} or \texttt{telemetry-mbed.h}
\end{itemize}

\subsection{Usage}
\begin{itemize}
  \item Instantiate both a telemetry HAL (again, for your platform) and a \texttt{Telemetry} object: \\
  Arduino example using \texttt{Serial1} as the UART pins:
  \begin{lstlisting}[language=C++]
telemetry::ArduinoHalInterface telemetry_hal(Serial1);
telemetry::Telemetry telemetry_obj(telemetry_hal);
  \end{lstlisting}  
  mBed example using \texttt{PTA2} as UART transmit and \texttt{PTA1} as UART receive:
  \begin{lstlisting}[language=C++]
MODSERIAL telemetry_serial(PTA2, PTA1);
telemetry::MbedHal telemetry_hal(telemetry_serial);
telemetry::Telemetry telemetry_obj(telemetry_hal);
  \end{lstlisting}
  \item Instantiate telemetry data objects:
\begin{lstlisting}[language=C++]
telemetry::Numeric<uint32_t> tele_time_ms(telemetry_obj, "time", "Time", "ms", 0);
telemetry::NumericArray<uint16_t, 128> tele_linescan(telemetry_obj, "linescan", "Linescan", "ADC", 0);  
telemetry::Numeric<float> tele_motor_pwm(telemetry_obj, "motor", "Motor PWM", "%DC", 0);
\end{lstlisting}
  The constructor signatures are:
  \begin{itemize}
    \item \texttt{template <typename T>\\ Numeric(Telemetry\& telemetry\_container, const char* internal\_name, const char* display\_name, const char* units, T init\_value)} \\
    \texttt{Numeric} describes numeric data of type \texttt{T}. Only 8-, 16- and 32-bit unsigned integers and single-precision floats are currently supported (but why would you use double-precision on a dinky embedded processor?!).
    \begin{itemize}
      \item \texttt{telemetry\_container}: a reference to a \texttt{Telemetry} object to associate this data with.
      \item \texttt{internal\_name}: a string giving this object an internal name to be referenced in code.
      \item \texttt{display\_name}: a string giving this object a human-friendly name.
      \item \texttt{units}: units this data record is in (not currently used, but may be useful for automation later).
      \item \texttt{init\_value}: initial value.
    \end{itemize}
    \item \texttt{template <typename T, uint32\_t array\_count>\\ NumericArray(Telemetry\& telemetry\_container, const char* internal\_name, const char* display\_name, const char* units, T elem\_init\_value)} \\
    \texttt{NumericArray} describes an array of numeric objects of type \texttt{T}. Same constraints apply as from Numeric. The array size \texttt{array\_count} is a template parameter (constant of sorts) to avoid dynamic memory allocations.
    \begin{itemize}
      \item \texttt{telemetry\_container}: a reference to a \texttt{Telemetry} object to associate this data with.
      \item \texttt{internal\_name}: a string giving this object an internal name to be referenced in code.
      \item \texttt{display\_name}: a string giving this object a human-friendly name.
      \item \texttt{units}: units this data record is in (not currently used, but may be useful for automation later).
      \item \texttt{elem\_init\_value}: initial value of array elements.
    \end{itemize}
  \end{itemize}
  \item You can optionally specify additional parameters on the data objects. The only one is the numerical value limits, which can be used to set manual plot bounds on a data visualizer.
\begin{lstlisting}[language=C++]
telemetry::Numeric<float> tele_motor_pwm(telemetry_obj, "motor", "Motor PWM", "%DC", 0);
tele_motor_pwm.set_limits(0, 1);
\end{lstlisting}
  Note that this (currently) does not affect the embedded code at all - the values will not be clipped nor will warnings / errors be thrown if an out-of-bounds value if set.
  \item Transmit the data definitions once at the beginning of your code, after you have finished defining all the data.
  \begin{lstlisting}[language=C++]
telemetry_obj.transmit_header();
  \end{lstlisting}
  \item Load data to be transmitted into the telemetry objects. \\
  These objects contain the code necessary to transmit and receive telemetry data, but otherwise behave similarly to their template types. For example, you can use the \texttt{tele\_linescan} object as an array:
\begin{lstlisting}[language=C++]
telemetry::NumericArray<uint16_t, 128> tele_linescan(telemetry_obj, "linescan", "Linescan", "ADC", 0);  
...
uint16_t* data = cam1.read();
for (uint16_t i=0; i<CAMERA_PIXEL_COUNT; i++) {
  tele_linescan[i] = data[i];
}
\end{lstlisting}
  Writing to the objects flags the new data to be transmitted on the next telemetry IO operation.
  \item Regularly call \texttt{Telemetry}'s \texttt{do\_io} operation to transmit any updated data and handle received data (like remote set commands).
\begin{lstlisting}[language=C++]
telemetry_obj.do_io();
\end{lstlisting}
  \textbf{Note that the code may not be thread-safe.} It scans through all the telemetry data objects, checking for and writing updated values to the output stream.
  \item Telemetry data objects may have their values set remotely, though they will not latch (i.e. setting it on the embedded-side causes the remotely-set value to be overwritten). As an example, if \texttt{tele\_motor\_pwm} is not set anywhere else, this code allows the motor PWM duty cycle to be set remotely.
\begin{lstlisting}[language=C++]
PwmOut MotorOutput(PTA4);
telemetry::Numeric<float> tele_motor_pwm(telemetry_obj, "motor", "Motor PWM", "%DC", 0);
...
telemetry_obj.do_io(); // remote set commands to tele_motor_pwm are processed here
MotorOutput.write(tele_motor_pwm);
\end{lstlisting}
  \item Note that you may continue to use the UART to transmit other data (like \texttt{printf}s) as long as this is not done during the middle of a \texttt{Telemetry} \texttt{do\_io} operation or contains the start-of-frame sequence (\texttt{0x05 0x39}).
\end{itemize}

\section{Receive-side Quick Start Guide}
All the receive-side code is contained in (repository base)\texttt{/client-py}.
\subsection{Plotter}
Run the plotter by going into (repository base)\texttt{/client-py} and running \texttt{python plotter.py} and passing in the serial port, baud rate (optional, defaults to 38400), independent variable (optional, defaults to \texttt{time}), and independent variable timespan (optional, defaults to 10000). The arguments list can be obtained through running \texttt{python plotter.py --help}

The plotter must be running when the header is transmitted, otherwise it will fail to decode the data packets (and notify you of such). The plotter will automatically reinitialize upon receiving a new header.

This simple plotter graphs all the data against a selected independent variable (like time). Numeric data is plotted as a line graph and array-numeric data is plotted as a waterfall / spectrograph-style graph in real-time. Regular UART data (like from \texttt{printf}s) is also written to the console in real-time. All received data, including from \texttt{printf}s, is logged to a CSV. A new CSV is created each time a new header packet is received, with a timestamped filename.

You can double-click a plot to inspect its latest value numerically and optionally remotely set it to a new value.

If you feel really adventurous, you can also try to mess with the code to plot things in different styles. For example, the plot instantiation function from a received header packet is in \texttt{subplots\_from\_header}. The default just creates a line plot for numeric data and a waterfall plot for array-numeric data. You can make it do fancier things, like overlay a numerical detected track position on the raw camera waterfall plot.

\subsection{Console Demo}
The console demo is also in (repository base)\texttt{/client-py} and can be run with \texttt{python console.py}. This simply prints the parsed received packets and can be useful if trying to learn the parser API.

\subsection{Parser Library}
The parser code is in (repository base)\texttt{/client-py/telemetry/parser.py}. No official documentation is currently available, but docstrings are scattered in important places and you can reference the console demo or plotter code.
\\
\\
That's all, folks!
\end{document}
